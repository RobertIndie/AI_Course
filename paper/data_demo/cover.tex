\thusetup{
  %******************************
  % 注意:
  %   1. 配置里面不要出现空行
  %   2. 不需要的配置信息可以删除
  %******************************
  %
  %=====
  % 秘级
  %=====
  secretlevel={秘密},
  secretyear={10},
  %
  %=========
  % 中文信息
  %=========
  ctitle={误差反向传播神经网络的Matlab实现及应用 \version},
  cdegree={工学硕士},
  cdepartment={计算机科学与技术系},
  cmajor={计算机科学与技术},
  cauthor={杨子棵},
  csupervisor={郑纬民教授},
  cassosupervisor={陈文光教授}, % 副指导老师
  ccosupervisor={某某某教授}, % 联合指导老师
  % 日期自动使用当前时间,若需指定按如下方式修改:
   cdate={超新星纪元},
  %
  % 博士后专有部分
  cfirstdiscipline={计算机科学与技术},
  cseconddiscipline={系统结构},
  postdoctordate={2009年7月——2011年7月},
  id={编号}, % 可以留空: id={},
  udc={UDC}, % 可以留空
  catalognumber={分类号}, % 可以留空
  %
  %=========
  % 英文信息
  %=========
  etitle={An Introduction to \LaTeX{} Thesis Template of Tsinghua University v\version},
  % 这块比较复杂,需要分情况讨论:
  % 1. 学术型硕士
  %    edegree:必须为Master of Arts或Master of Science(注意大小写)
  %             “哲学、文学、历史学、法学、教育学、艺术学门类,公共管理学科
  %              填写Master of Arts,其它填写Master of Science”
  %    emajor:“获得一级学科授权的学科填写一级学科名称,其它填写二级学科名称”
  % 2. 专业型硕士
  %    edegree:“填写专业学位英文名称全称”
  %    emajor:“工程硕士填写工程领域,其它专业学位不填写此项”
  % 3. 学术型博士
  %    edegree:Doctor of Philosophy(注意大小写)
  %    emajor:“获得一级学科授权的学科填写一级学科名称,其它填写二级学科名称”
  % 4. 专业型博士
  %    edegree:“填写专业学位英文名称全称”
  %    emajor:不填写此项
  edegree={Doctor of Engineering},
  emajor={Computer Science and Technology},
  eauthor={Xue Ruini},
  esupervisor={Professor Zheng Weimin},
  eassosupervisor={Chen Wenguang},
  % 日期自动生成,若需指定按如下方式修改:
  % edate={December, 2005}
  %
  % 关键词用“英文逗号”分割
  ckeywords={误差反向传播神经网络, 训练, 测试, Matlab, 分类问题},
  ekeywords={Error Back Propagation Neural Network, Training, Testing, Matlab, Classification problem}
}

% 定义中英文摘要和关键字
\begin{cabstract}
误差反向传播神经网络是一种典型的人工神经网络,在一些数学问题的解决中起到了重要的作用。本文介绍了误差反向传播神经网络的原理和模型,对网络的前向传播和后向传播进行了详细的数学推导,并利用Matlab对网络进行了建模,通过分类问题对所得的模型进行训练、测试、分析,测试结果的正确率均在$99\%$以上,最后对所构建的模型和测试情况进行了简要说明和分析。

本文的创新点主要有:
  \begin{itemize}
  	\item 对误差反向传播神经网络进行了数学推导;
    \item 用Matlab实现误差反向传播神经网络;
    \item 用误差反向传播神经网络解决分类问题。
  \end{itemize}

\end{cabstract}

% 如果习惯关键字跟在摘要文字后面,可以用直接命令来设置,如下:
% \ckeywords{\TeX, \LaTeX, CJK, 模板, 论文}

\begin{eabstract}
  The Error Back Propagation Neural Network is a typical artificial neural network and plays an important role in solving some mathematical problems. This paper introduces the principle and model of the Error Back Propagation Neural Network, and makes a detailed mathematical derivation of the forward and backward propagation of the network, and uses Matlab to model the network, through classification problem. The classification problem is to train, test, and analyze the obtained model. The correct rate of the test results is above $99\%$. Finally, the model and test conditions are briefly described and analyzed.
  
  The main innovations of this paper are:
    \begin{itemize}
  	\item Mathematical Derivation of Error Back Propagation Neural Networks;
  	\item Model Error Back Propagation Neural Network in Matlab;
  	\item Using Error Back Propagation Neural Network to Solve Classification Problems。
  \end{itemize}
\end{eabstract}

% \ekeywords{\TeX, \LaTeX, CJK, template, thesis}
