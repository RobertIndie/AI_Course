\chapter{误差反向传播神经网络概述}
\label{cha:intro}

在机器学习和认知科学领域中,人工神经网络是一种模仿生物神经网络(动物的中枢神经系统,特别是大脑)的结构和功能的数学模型或计算模型,用于对函数进行估计或近似。神经网络由大量的人工神经元联结进行计算。大多数情况下人工神经网络能在外界信息的基础上改变内部结构,是一种自适应系统。

误差反向传播神经网络\cite{RepresentationsByBP}是一种与最优化方法(如梯度下降法)结合使用的,用来训练人工神经网络的常见方法。该方法对网络中所有权重计算损失函数的梯度。这个梯度会反馈给最优化方法,用来更新权值以最小化损失函数。反向传播要求有对每个输入值想得到的已知输出,来计算损失函数梯度。因此,它通常被认为是一种监督式学习方法,虽然它也用在一些无监督网络中。

误差反向传播神经网络的算法思想是 :让信号正向传播和误差反向传播这两个过程交替循环进行,信号每正向传播一次之后都要计算一次误差,让误差沿梯度负方向下降一个很小的变化量,将这个误差变化量反向传播到神经网络的各层,然后对各层参数矩阵的值进行调整,之后再进行下一次循环。理论上讲,经过多次循环之后,神经网络的误差会收敛到一个较为稳定的范围之内,这时候就可以认为各层参数矩阵的值达到了理想值,即模型达到了最优状态。这种误差反向传播训练算法是人工神经网络的核心思想,当前比较流行的卷积神经网络、循环神经网络等训练算法都是在误差反向传播神经网络算法基础上经过优化和改进发展而来的。\cite{BPNNPrinciple}

