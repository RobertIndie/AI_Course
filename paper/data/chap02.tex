\chapter{网络模型与原理}
\label{cha:algorithm}

\section{模型概述}
在神经网络信号的正向传播中,神经元接收到来自其他神经元的输入信号,这些信号乘以权重累加到神经元接收的总输入值上,随后与当前神经元的阈值进行比较,然后通过激活函数处理,产生神经元的输出。理想的激活函数是阶跃函数,然而阶跃函数的缺点是不连续,不可导,所以常用$sigmoid$函数作为误差反向传播神经网络的激活函数。

误差反向传播神经网络的核心思想就是:通过调整各神经元之间的权值,将误差由隐含层向输入层逐层反传,也就是先实现信号的正向传播到误差的反向传播过程。所以误差反向传播神经网络算法的核心步骤如下 :(1)求得在特定输入下实际输出与理想输出的平方误差函数(误差函数或者叫代价函数);(2)利用误差函数对神经网络中的阈值以及连接权值进行求导;(3)根据梯度下降算法,对极小值进行逼近,当满足条件时,跳出循环。

在本篇论文中,将使用的是有监督的误差反向传播神经网络。故我们需要提供一个包括输入和期望输出的训练集,以在训练的过程中获得它相对于训练集的误差。正向传播 :输入样本—输入层—各隐含层—输出层 ;若输出层的实际输出与期望的输出不符,则误差反传 :误差表示—修正各层神经元的权值 ;直到网络输出的误差减少到可以接受的程度,或者进行到预先设定的学习次数为止。

\section{模型推导}

在反向传播神经网络中,通常选择的是$sigmoid$函数作为激活函数,这类函数的导数有一个很好的性质,就是自身相关。$sigmoid$函数如式~(\ref{equ:algorithm:sigmoid})。
\begin{equation}
\label{equ:algorithm:sigmoid}
sigmoid(x) = \frac{1}{1+e^{-x}}
\end{equation}
其导数如式~~(\ref{equ:algorithm:sigmoidDerivative})
\begin{equation}
\label{equ:algorithm:sigmoidDerivative}
sigmoid'(x) = sigmoid(x)(1-sigmoid(x))
\end{equation}
假定现有一个四层的神经网络,包括一层输入层、两层隐含层和一层输出层。假定各层的神经元数量分别为:6,4,3,2。接下来对各层的输入和输出变量进行定义,输入至输入层的输入向量如式~(\ref{equ:algorithm:input0});对应的第一层隐含层的输入向量如式~(\ref{equ:algorithm:input0});第二层隐含层的输入向量为式~(\ref{equ:algorithm:input0});输出层的输入向量为式~(\ref{equ:algorithm:input0})\cite{BPNNPrincipleY}
\begin{equation}
\label{equ:algorithm:input0}
Y^{(0)} = [x_1,x_2,\cdots,x_6]^T =[a_1(0),a_2(0),a_3(0),a_4(0),a_5(0),a_6(0)]^T
\end{equation}
\begin{equation}
\label{equ:algorithm:input1}
Y^{(1)} = [a_1(1),a_2(1),a_3(1),a_4(1)]^T 
\end{equation}
\begin{equation}
\label{equ:algorithm:input2}
Y^{(2)} = [a_1(2),a_2(2),a_3(2)]^T
\end{equation}
\begin{equation}
\label{equ:algorithm:input3}
Y^{(3)} = [a_1(3),a_2(3)]^T
\end{equation}
假定第$n$层($n>=2$)神经元的个数为$i$个,第$n-1$层的神经元个数为$j$个,定义第$n$层与第$n-1$层之间的连接权值矩阵如式~(\ref{equ:algorithm:wMatrix})
\begin{equation}
\label{equ:algorithm:wMatrix}
W^{(n)}=\begin{bmatrix}
w_{11}^{(n)} & \cdots & w_{1j}^{(n)} \\
\vdots & \cdots & \vdots \\
w_{i1}^{(n)}& \cdots & w_{ij}^{(n)}
\end{bmatrix}
\end{equation}
定义第$n$层神经元的阈值向量如式~(\ref{equ:algorithm:thetaVector})
\begin{equation}
\label{equ:algorithm:thetaVector}
\theta^{(n)} = [\theta^{(n)}_{(1)},\theta^{(n)}_{(2)},...,\theta^{(n)}_{(n)}]^T 
\end{equation}
则第$n$层神经元的输出矩阵如式~(\ref{equ:algorithm:vMatrix})
\begin{equation}
\label{equ:algorithm:vMatrix}
V^{(n)}=W^{(n)}Y^{(n)}+\theta^{n}
\end{equation}
第$n$层的输入与第$n-1$层的输出的关系如~(\ref{equ:algorithm:VAndY})
\begin{equation}
\label{equ:algorithm:VAndY}
Y^{(n)}=f(V^{(n-1)})
\end{equation}
其中,$f$为激活函数,常使用$sigmoid$函数作为激活函数,故如式~(\ref{equ:algorithm:VAndY_sigmoid})
\begin{equation}
\label{equ:algorithm:VAndY_sigmoid}
Y^{(n)}=sigmoid(V^{(n-1)})
\end{equation}
目标是希望通过调整$W$使得输出和目标的误差最小,利用最小二乘思想:
\begin{equation}
E = \frac{1}{k}\sum_{l=1}^k(T_l-Y_l^{(n)})^2 =  \frac{1}{k}\sum_{l=1}^k(\delta_l^{(n)})^2
\end{equation}
其中,$\delta_l$如式~(\ref{equ:algorithm:delta})
\begin{equation}
\label{equ:algorithm:delta}
\delta_l = (T_l-Y_l^{(n)})
\end{equation}
因为误差反向传播神经网络是反馈式网络,所以在更新权值时是从后向前更新的,首先更新的是最后一层的权值。所以有:
\begin{equation}
E = \frac{1}{k}\sum_{l=1}^k(T_l-Y_l^{(n)})^2 \\
= \frac{1}{k}\sum_{l=1}^k(T_l-f(V_l^{(n)}) )^2 \\
= \frac{1}{k}\sum_{l=1}^k(T_l-\sum_{i=1}^{q}W_{li}^{q}Y_i^{(n-1)})^2
\end{equation}
因此$E$和$W$的求导关系为:
\begin{equation}
\frac{\partial E}{\partial W_{li}^{(n)}} = \frac{\partial E}{\partial Y_l^{(n)}}\frac{\partial Y_l^{(n)}}{\partial V_l^{(n)}}\frac{\partial V_l^{(n)}}{\partial W_{li}^{(n)}}
\end{equation}
其中:
\begin{equation}
\label{equ:algorithm:1}
\frac{\partial E}{\partial Y_l^{(n)}} = (Y_l-T_l) = -\delta_l 
\end{equation}
\begin{equation}
\label{equ:algorithm:2}
\frac{\partial Y_l^{(n)}}{\partial V_l^{(n)}} = f^{'(n)}(V_l^{(n)})
\end{equation}
\begin{equation}
\label{equ:algorithm:3}
\frac{\partial V_l^{(n)}}{\partial W_{li}^{(n)}} = Y_i^{(n-1)}
\end{equation}
根据式~(\ref{equ:algorithm:1})、式~(\ref{equ:algorithm:2})、式~(\ref{equ:algorithm:3})可得:
\begin{equation}
\frac{\partial E}{\partial W_{li}^{(n)}}=-\delta_l f^{'(n)}(V_l^{(n)})  Y_i^{(n-1)}
\end{equation}
令
\begin{equation}
S^{(n)}=\delta_l f^{'(n)}(V_l^{(n)}) 
\end{equation}
故可得:
\begin{equation}
\label{key}
\frac{\partial E}{\partial \theta^{(n)}}=\frac{\partial E}{\partial V^{(n)}}=-S^{(n)}
\end{equation}

\section{算法流程}
接下来阐述误差反向传播神经网络的算法流程。第一步随机初始化各层神经元之间的连接权值矩阵;第二步通过式~(\ref{equ:algorithm:vMatrix})计算每层神经元的输出矩阵;第三步求各层神经元的误差:
\begin{equation}
\delta =Y^{(n)}-T
\end{equation}
第四步对于输出层求得反向传播误差因子:
\begin{equation}
S^{(n)}=f(V^{(n)})\delta
\end{equation}
对于输出层以前的各层求得反向传播误差因子:
\begin{equation}
S^{(n)}=f'^{(n)}(V^{(n)})W^{(n+1)^T}S^{(n+1)}
\end{equation}
第五步更新各神经元层与上一层的连接权值矩阵:
\begin{equation}
W_{lk}^{(n)}=W_{lk}^{(n-1)}+\eta S_k^{(n)}Y_l^{(n-1)}
\end{equation}
\begin{equation}
\theta_k^{(n)}=\theta_k^{(n-1)}+\eta S_k^{(n)}
\end{equation}
判断是否达到条件,如是否达到期望正确率,如未达到条件则返回第二步。

