\chapter{网络的Matlab实现}
\label{cha:matlabCode}

依据上一章节推导得出的算法原理和流程,我们利用Matlab来实现误差方向传播神经网络。\footnote{本文所涉及的所有源码均已放在Github上,地址为:https://github.com/RobertIndie/BPNN}

首先,我们需要创建初始化神经元层间的连接权值矩阵的函数:

\indent
\begin{lstlisting}
function [ W ] = init_w( levels )
% 初始化连接权值矩阵,返回每一层之间的连接权值矩阵
    n_level = numel(levels) - 1;
    W = cell(n_level,1);
    for i=1:n_level
       W{i} = rand(levels(i+1),levels(i));
    end
end
\end{lstlisting}

接下来创建初始化每层神经元的阈值矩阵的函数:

\begin{lstlisting}
function [ theta ] = init_theta( levels )
% 初始化阈值矩阵
    n_level = numel(levels) - 1;
    theta = cell(n_level, 1);
    for i=1:n_level
       theta{i} = rands(levels(i+1),1); 
    end
end
\end{lstlisting}

接下来创建激活函数,这里我们使用$sigmoid$作为神经网络的激活函数:

\begin{lstlisting}
function [ y ] = sigmoid( x )
% sigmoid 函数
    y = 1./(1+exp(-x));
end
\end{lstlisting}

接下来创建神经网络前向传播运算函数:

\begin{lstlisting}
function [ output,Y] = predict( input, W, theta)
% 前向传播
% input: 输入向量
% W: 连接权值矩阵元胞数组
% theta: 阈值矩阵
% output: 输出向量
% Y: 各层的输出
    f = @sigmoid;
    n_w = numel(W);
    Y = cell(n_w+1, 1);
    y = input';
    Y{1}=input';
    for i=1:n_w
        V = W{i}*y+theta{i};
        y = f(V);
        Y{i+1} = y;
    end
    output = y;
end

\end{lstlisting}

下面是训练神经网络的函数:

\begin{lstlisting}
function [ W,theta,record ] = train( X, levels, step, min_err, computeErrorFunc )
% 训练
% X: 训练集
% levels: 神经网络结构
% step: 趋近学习步长
% min_err: 期望错误率
% computeErrorFunc: 错误率计算函数
    r_i = 0;record = zeros(1,10000);% 历史错误率
    n_levels = size(levels, 2) - 1;
    n_input = levels(1);
    W = init_w(levels);
    theta = init_theta(levels);
    [n_data,col] = size(X);
    train_data = X(:,1:n_input); 
    label = X(:,n_input+1:end)';
    f = @sigmoid;
    while true
        for k=1:n_data
           [output, Y] =  predict(train_data(k,:), W, theta);
           delta = label(:,k) - output;
           % 从最后一层往前更新W
           for l=n_levels:-1:1
              net = W{l}*Y{l} + theta{l};
              if l == n_levels
                  S = diag(f(net).*(1-f(net)))*delta;
              else
                  S = diag((f(net).*(1-f(net))))*W{l+1}'*S;
              end
              new_W{l} = W{l} + step*S*Y{l}';
              new_theta{l} = theta{l} + step*S;
           end 
           W = new_W;
           theta = new_theta;
        end 

        y = predict(train_data, W, theta);
        error = computeErrorFunc(y, label)
        r_i = r_i + 1;
        record(r_i) = error; % 记录历史错误率
        if error < min_err
            break;
        end
    end
    record=record(1,1:r_i);
end
\end{lstlisting}

训练的中止条件为达到某一期望的最大错误率,错误率计算函数由函数参数给出。